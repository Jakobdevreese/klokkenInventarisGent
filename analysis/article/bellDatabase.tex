% ============================================================= %
%  --- Bell Database Article ---
% ============================================================= %

\documentclass[11pt, a4paper]{article}

% ------------------------------------------------------------- 
% Packages
% -------------------------------------------------------------
\usepackage{microtype} % Better typography
\usepackage{fontspec} % For loading fonts
\usepackage{fontawesome5} % For icons
\usepackage{parskip} % No paragraph indentation
\usepackage{ragged2e} % For better justification control
\usepackage{hyphenat} % For hyphenation control
\usepackage[left=2cm,right=2cm,top=2cm,bottom=2cm]{geometry} % Page margins
\usepackage{pgfplots} % For plots
\usepackage{pgf-pie} % For pie charts
\usepackage{amsmath} % For math
\usepackage{caption} % For captions


% -------------------------------------------------------------
% Font
% -------------------------------------------------------------
\usepackage{fontspec}
\setmainfont{Fira Code}


% -------------------------------------------------------------
% Styles
% -------------------------------------------------------------
\renewcommand{\abstractname}{} % Remove abstract title
\newlength{\abstractwidth}
\setlength{\abstractwidth}{0.8\textwidth}

% -------------------------------------------------------------
% Codeblocks
% -------------------------------------------------------------
\usepackage{listings} % For codeblocks
\usepackage{xcolor} % For colors
\definecolor{lgray}{gray}{0.9} % Light gray color

\lstdefinelanguage{CSS}{
  keywords={
    color, background, font, margin, padding, border, display, position,
    width, height, top, left, right, bottom, float, clear, overflow,
    text-align, vertical-align, line-height, letter-spacing, word-spacing,
    text-decoration, text-transform, font-family, font-size, font-weight,
    font-style, list-style, list-style-type, list-style-position,
    table-layout, border-collapse, border-spacing, caption-side, empty-cells,
    content, cursor, outline, visibility, z-index, zoom, filter, opacity,
    transition, animation, transform, box-shadow, text-shadow, border-radius,
    background-image, background-repeat, background-position, background-size,
    @media, @keyframes, !important
  },
  keywordstyle=\color{magenta},
  comment=[l]{//},
  morecomment=[s]{/*}{*/},
  commentstyle=\color{green},
  stringstyle=\color{purple},
  sensitive=true,
  morestring=[b]",
  morestring=[b]',
}

\lstdefinestyle{Pythonstyle}{
    language=Python,
    backgroundcolor=\color{lgray},
    commentstyle=\color{green},
    keywordstyle=\color{magenta},
    numberstyle=\tiny\color{gray},
    stringstyle=\color{purple},
    basicstyle=\ttfamily\footnotesize,
    breakatwhitespace=false,
    breaklines=true,
    captionpos=b,
    keepspaces=true,
    numbers=left,
    numbersep=5pt,
    showspaces=false,
    showstringspaces=false,
    showtabs=false,
    tabsize=2
}

\lstdefinestyle{CSSstyle}{
    language=CSS,
    backgroundcolor=\color{lgray},
    commentstyle=\color{green},
    keywordstyle=\color{magenta},
    numberstyle=\tiny\color{gray},
    stringstyle=\color{purple},
    basicstyle=\ttfamily\footnotesize,
    breakatwhitespace=false,
    breaklines=true,
    captionpos=b,
    keepspaces=true,
    numbers=left,
    numbersep=5pt,
    showspaces=false,
    showstringspaces=false,
    showtabs=false,
    tabsize=2
}

\lstdefinestyle{HTMLstyle}{
    language=HTML,
    backgroundcolor=\color{lgray},
    commentstyle=\color{green},
    keywordstyle=\color{magenta},
    numberstyle=\tiny\color{gray},
    stringstyle=\color{purple},
    basicstyle=\ttfamily\footnotesize,
    breakatwhitespace=false,
    breaklines=true,
    captionpos=b,
    keepspaces=true,
    numbers=left,
    numbersep=5pt,
    showspaces=false,
    showstringspaces=false,
    showtabs=false,
    tabsize=2
}


% ============================================================== %
%  --- Document Information ---
% ============================================================== %

\title{\Huge Structuring Bell Heritage: A Comprehensive Database Schema and Framework for Carillons and Bells}
\author{\LARGE{Jakob De Vreese} \\ \texttt{\small{jakobdevreese@gmail.com}}}
\date{March 2025}

% ============================================================== %
%  --- Document ---
% ============================================================== %

\begin{document}

% -------------------------------------------------------------
% Title Page
% -------------------------------------------------------------
\begin{titlepage}
    \newgeometry{top=6cm}
    \maketitle
    \thispagestyle{empty}
    \vspace{1cm}
    \begin{center}
        \small{last updated: \today}
    \end{center}
    \vspace{2cm}
    
    \begin{center}
        \rule{\textwidth}{0.4pt}
        \vspace{1em}
        
        \begin{minipage}{\abstractwidth}
            \setlength{\rightskip}{0pt plus 1fil} % Allow extra stretch on each line
            \justifying
            \noindent
            This research addresses a significant challenge in campanology: while enthusiasts and professionals 
            gather extensive data on bells and carillons, this valuable information often remains fragmented, 
            inconsistently structured, and inaccessible to the wider community. We propose the development of a 
            standardized framework for campanological data management — a comprehensive database schema designed 
            to accommodate the documentation needs for bells, carillons, and related heritage objects. This framework 
            aims to establish a foundation that balances standardization with flexibility, enabling individual 
            researchers and institutions to maintain autonomous databases while adhering to compatible structural 
            principles. The project consists of two key deliverables: first, a detailed entity-relationship model 
            that defines core data elements and their relationships; and second, an open-source Django web application 
            that implements this schema, providing accessible interfaces for data entry, management, and retrieval. 
            By establishing this standardized yet adaptable framework, we facilitate the potential integration of 
            distributed campanological datasets, thereby enhancing opportunities for comprehensive analysis, heritage 
            preservation, and collaborative research across both professional and amateur campanological communities.
          \end{minipage}
        
        \vspace{1em}
        \rule{\textwidth}{0.4pt}
    \end{center}

    \restoregeometry
\end{titlepage}

% -------------------------------------------------------------
% Table of Contents
% -------------------------------------------------------------

\clearpage
\setcounter{page}{1}
\pagenumbering{arabic}
\tableofcontents
\clearpage

% -------------------------------------------------------------
% Begin of the document
% -------------------------------------------------------------

\section{Introduction}

In the field of campanology, a wealth of valuable data is continuously gathered 
by both professional researchers and dedicated enthusiasts. This information 
encompasses a diverse range of attributes: physical dimensions of bells, acoustic 
properties, historical provenance, inscriptions, decorative elements, mechanical 
configurations, and the compositional arrangements of bell sets within carillons 
and towers. Despite this rich accumulation of knowledge, the campanological field 
faces a significant challenge: the lack of standardized methods for structuring, 
storing, and sharing this information in a consistent and accessible manner.

The current landscape of campanological data management is characterized by fragmentation 
and inconsistency. Individual researchers, institutions, and bell enthusiasts typically 
develop their own documentation methods, resulting in a patchwork of incompatible datasets. 
These range from handwritten notes and spreadsheets to custom-built databases with idiosyncratic 
schemas. While these approaches may serve immediate project needs, they create substantial 
barriers to knowledge sharing, comparative analysis, and comprehensive research across the broader field.

This fragmentation presents several critical limitations:

\begin{itemize}
    \item \textbf{Restricted Accessibility:} \\
        Valuable research remains siloed within individual projects or institutions, limiting potential insights from combined datasets.
    \item \textbf{Duplication of Effort:} \\
        Researchers frequently redocument the same bells due to lack of awareness or access to existing documentation.
    \item \textbf{Inconsistent Terminology:} \\
        The absence of standardized vocabulary and measurement protocols creates difficulties in comparing data between sources.
    \item \textbf{Limited Analytical Capacity:} \\
        The inability to easily aggregate distributed datasets hinders comprehensive statistical analysis and pattern recognition across larger samples.
\end{itemize}

This paper presents a solution to these challenges through the development of a comprehensive database schema and 
framework specifically designed for campanological research. Our approach balances the need for standardization 
with the flexibility required to accommodate the diverse documentation needs across the field. We propose both a 
conceptual data model — represented as an entity-relationship diagram — and its practical implementation as an 
open-source web application built using the Django framework.

The primary objectives of this project are to:
\begin{enumerate}
    \item Design and propose a flexibel, comprehensive and adaptable database schema for campanological data management.
    \item Develop an open-source, easy to use, web application that implements this schema.
    \item Facilitate the integration of distributed campanological datasets, enabling comprehensive analysis and collaborative research.
\end{enumerate}

\subsection{Itemize}

\begin{itemize}
    \item \textbf{Item 1:} \\
        Some information about item 1.
    \item \textbf{Item 2:} \\
        Some information about item 2.
    \item \textbf{Code: } \\
        This template has styles for \textbf{Python}, \textbf{CSS}, and \textbf{HTML}.
\end{itemize}
    
\section{Database Schema}

When designing a database schema for campanological data, we must consider the diverse range of attributes.
We start with the smallest unit we want to document: the bell. A bell has a set of physical properties, such absence
diamieter, height, weight and inscriptions, but also has a set of relationships with other entities, like bell founder. 
A bell hangs most of the time in a tower, wich gives away the geographical location, and can be part of a larger entity, the carillon. 
When we take this into account, we can start designing the database schema.

\begin{figure}[h!]
    \centering
    \includegraphics[width=0.8\textwidth]{images/basic_entities.png}
    \caption{Basic entities in the campanological database schema.}
    \label{fig:basic-entities}
\end{figure}

When we look at the basic entities in the campanological database schema, we see that the bell is the central entity.
When we put this central in our database, and we add the relationships between the entities, we get a readable and 
connectible database. To lay the basis for a standardization in this field, we off course need more. We want a minimum
of fields that every database in our field should have, and a standardized way of taking basic field measurements.

\subsection{Bell}

The bell is our starting point. The minimal attributes needed to describe a bell, in a way that it can't be confused with another bell, are:

\begin{itemize}
    \item \textbf{Founder:} \\
        The founder of a bell, this can be a person or a company. (Other enitie)
    \item \textbf{Weight:} \\
        The weight of the bell (in kilograms)
    \item \textbf{Pitch:} \\
        The pitch of the bell (in Hertz)
    \item \textbf{Diameter:} \\
        The diameter of the bell (in centimeters) - TODO - add more
    \item \textbf{Height:} \\
        The height of the bell (in centimeters) - TODO - add more
    \item \textbf{Inscriptions:} \\
        The inscriptions on the bell in words
    \item \textbf{Year:} \\
        The year the bell was cast.
    \item \textbf{Location:} \\
        The location of the bell. (Other entity) - TODO - add more

Other attributes can be added to the bell entity, but these are the minimal attributes needed to describe a bell. The researchers should strive 
to document these attributes as complete as possible for all bells.

\subsection{Founder}

A founder is a person or a company that casted the bell. This is a tricky entity, because a founder can be a person or a company, and can have multiple
ways to write the name, or there are founders that worked with other companies. How we propose to solve this in the standardization is to have the attributes
in the database for both the name of the founder, as well as the company, and the years the company or the founder was active. This way, when we have a founder 
that worked with multiple companies, or companies that had muliple founders that where named by name on the bell, we can document this, but they will come up 
when we search for the founder.

    

\end{document}